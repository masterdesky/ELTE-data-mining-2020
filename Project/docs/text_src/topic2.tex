\section{Bound and scattering states in central potentials}
\begin{center}
\textit{derivation of the radial Schrödinger equation; bound states in a central potential; the spatial, spin, and isospin components of the two-nucleon wave function, and their symmetry properties; list of possible two-nucleon states}
\end{center}

\subsection{Two-nucleon Schrödinger equation}
To study the bound and scattering states of the deuteron, we need to solve the two-nucleon Schrödinger equation

\begin{equation} \label{eq:2.1}
	\hat{H} \Ket{\Psi}
	=
	E \Ket{\Psi}
\end{equation}
Where the $\hat{H}$ Hamiltonian could be phrased as the sum of the kinetic part of the individual Hamiltonians of the two particles, and an additional potential. Thus the two-nucleon Schrödinger equation looks like the following:

\begin{equation} \label{eq:2.2}
	\left[
		- \frac{\hbar^{2}}{2 m_{1}} \Delta_{1}
		- \frac{\hbar^{2}}{2 m_{2}} \Delta_{2}
		+ V \left( \vec{r}_{1}, \vec{r}_{2} \right)
	\right]
	\Psi \left( \vec{r}_{1}, \vec{r}_{2} \right)
	=
	E_{\text{tot}} \Psi \left( \vec{r}_{1}, \vec{r}_{2} \right)
\end{equation}
Since the Hamiltonian is invariant under spatial translation the potential is dependent of the relative coordinate of the two particles. Introducing the $\boldsymbol{r}$ relative and $\boldsymbol{R}$ barycentric coordinates where

\begin{equation} \label{eq:2.3}
	\boldsymbol{r} = \boldsymbol{r}_{2} - \boldsymbol{r}_{1}
\end{equation}
\begin{equation} \label{eq:2.4}
	\boldsymbol{R} = \frac{m_{1} \boldsymbol{r}_{1} + m_{2} \boldsymbol{r}_{2}}{m_{1} + m_{2}}
\end{equation}
also using the chaining rules for derivatives

\begin{equation} \label{eq:2.5}
	\frac{\partial}{\partial \vec{r}_{1}}
	=
	\frac{\partial}{\partial \vec{R}} \frac{\partial \vec{R}}{\partial \vec{r}_{1}}
	+
	\frac{\partial}{\partial \vec{r}} \frac{\partial \vec{r}}{\partial \vec{r}_{1}}
	=
	\frac{\partial}{\partial \vec{R}} \frac{m_{1}}{M}
	+
	\frac{\partial}{\partial \vec{r}} * \left( -1 \right)
	=
	\frac{m_{1}}{M} \frac{\partial}{\partial \vec{R}}
	-
	\frac{\partial}{\partial \vec{r}}
\end{equation}
\begin{equation} \label{eq:2.6}
	\frac{\partial}{\partial \vec{r}_{2}}
	=
	\frac{\partial}{\partial \vec{R}} \frac{\partial \vec{R}}{\partial \vec{r}_{2}}
	+
	\frac{\partial}{\partial \vec{r}} \frac{\partial \vec{r}}{\partial \vec{r}_{2}}
	=
	\frac{\partial}{\partial \vec{R}} \frac{m_{2}}{M}
	+
	\frac{\partial}{\partial \vec{r}} * 1
	=
	\frac{m_{2}}{M}	\frac{\partial}{\partial \vec{R}}
	+
	\frac{\partial}{\partial \vec{r}}
\end{equation}
causes the kinetic part of the Hamiltonian to decompose into two parts:

\begin{equation} \label{eq:2.7}
	\hat{H}_{\text{kin}}
	\equiv
	\frac{\hbar^{2}}{2M} \Delta_{\boldsymbol{R}} + \frac{\hbar^{2}}{2 \mu} \Delta_{\boldsymbol{r}} 
\end{equation}
Where $\mu$ is the relative mass

\begin{equation} \label{eq:2.8}
	\mu
	=
	\frac{m_{1} m_{2}}{m_{1} + m_{2}}
\end{equation}
Let us separate the $\Psi \left( \vec{r}_{1}, \vec{r}_{2} \right)$ wave function into the product of an $\boldsymbol{R}$ and an $\boldsymbol{r}$ depending part respectively:

\begin{equation} \label{eq:2.9}
	\Psi \left( \vec{r}_{1}, \vec{r}_{2} \right)
	\equiv
	\phi \left( \boldsymbol{R} \right) * \psi \left( \boldsymbol{r} \right)
\end{equation}
In this case the differential equation in Eq. \eqref{eq:2.2} decomposes into two separate parts:

\begin{equation} \label{eq:2.10}
	- \frac{\hbar^{2}}{2M} \Delta \phi \left( \boldsymbol{R} \right)
	=
	E_{\text{CM}} \phi \left( \boldsymbol{R} \right)
	\quad , \quad
	\left( - \frac{\hbar^{2}}{2 \mu} \Delta + V \left( \boldsymbol{r} \right) \right) \psi \left( \boldsymbol{r} \right)
	=
	E \psi \left( \boldsymbol{r} \right)
\end{equation}
Where

\begin{equation} \label{eq:2.11}
	E_{\text{CM}} = E_{\text{tot}} - E
\end{equation}
is the energy of the center of mass. If $E < 0$ we speak about a bounded state and if $E > 0$ it is a scattering state. \par
In this case we would like to solve the (relative) Schrödinger equation on the right in Eq. \eqref{eq:2.10}. It is advisable to do this in a polar coordinate system using the spherical Laplace operator declared in Eq. \eqref{eq:1.28}. The equation becomes the following:

\begin{equation} \label{eq:2.12}
	- \frac{\hbar^{2}}{2 \mu}
	\left[
		\frac{1}{r} \frac{\partial^{2}}{\partial r^{2}} \left( r \Psi \left( \boldsymbol{r} \right) \right)
		+
		\frac{\hat{L}^{2}}{r^{2}} \Psi \left( \boldsymbol{r} \right)
	\right]
	+
	V \left( \boldsymbol{r} \right)
	=
	E \Psi \left( \boldsymbol{r} \right)
\end{equation}
Where the $(\theta, \varphi)$-dependent part of the spherical Laplacian (multiplied out by the factor $\frac{1}{r^{2}}$) is marked with $\hat{L}^{2}$. Using the method of partial wave analysis, the (relative) wave function may be expanded in spherical harmonics:

\begin{equation} \label{eq:2.13}
	\Psi \left( \boldsymbol{r} \right)
	=
	\sum_{l\,m} \frac{u_{l} \left( r \right)}{r} Y_{l}^{m} \left( \theta, \varphi \right)
\end{equation}
Thus we need to solve the differential equation

\begin{equation} \label{eq:2.14}
	- \frac{\hbar^{2}}{2 \mu}
	\left(
		u_{l}''
		-
		\frac{l \left( l + 1 \right)}{r^{2}}
	\right)
	+
	V \left( r \right) u_{l}
	=
	E u_{l}
\end{equation}
with the boundary condition $u_{l} \left( r = 0 \right) = 0$.

\subsection{Bound states}
Let us consider a square potential well to model the interaction between the nucleons with $b$ width and a depth of $- V_{0}$, where $V_{0} > 0$:

\begin{equation} \label{eq:2.15}
	V \left( r \right)
	=
	\begin{cases}
		- V_{0} & \text{if } r < b \\
		0       & \text{if } r > b \\
	\end{cases}
\end{equation}
First examine the $l = 0$ case. Since we're talking about bound states, $E < 0$ and $\left| E \right| < V_{0}$ by definition. Separate the the differential equation into two cases. The first one indicated by the Roman numeral I., where $r < b$ and the second, indicated by the numeral II., where $r > b$. In the second case, the potential is zero, so the two equations are the following:

\begin{equation} \label{eq:2.16}
	- \frac{\hbar^{2}}{2 \mu} u_{\text{I.}}'' - V_{0} u_{\text{I.}} = E u_{\text{I.}}
\end{equation}
\begin{equation} \label{eq:2.17}
	- \frac{\hbar^{2}}{2 \mu} u_{\text{II.}}'' = E u_{\text{II.}}
\end{equation}
We can rearrange the equations and introduce two substitution values:

\begin{equation} \label{eq:2.18}
	u_{\text{I.}}''
	=
	\underbrace{- \frac{2 \mu}{\hbar^{2}} \left( V_{0} + E \right)}_{\equiv -\kappa^{2}} u_{\text{I.}}
	\equiv
	- \kappa^{2} u_{\text{I.}}
\end{equation}
\begin{equation} \label{eq:2.19}
	u_{\text{II.}}''
	=
	\underbrace{- \frac{2 \mu}{\hbar^{2}} E}_{\equiv -k^{2}} u_{\text{II.}}
	\equiv
	- k^{2} u_{\text{II.}}
\end{equation}
Now we can make an Ansatz for both of the solutions:

\begin{equation} \label{eq:2.20}
	u_{\text{I.}}
	=
	A \sin \left( \kappa r \right)
	+
	B \cos \left( \kappa r \right)
\end{equation}
\begin{equation} \label{eq:2.21}
	u_{\text{II.}}
	=
	C e^{-kr}
	+
	D e^{kr}
\end{equation}
Because at $r \to \infty$ the $u_{\text{II.}} = 0$ and at $r = 0$ the $u_{\text{I.}} = 0$, the Ansatzes should be reshaped to satisfy this condition:

\begin{equation} \label{eq:2.22}
	u_{\text{I.}}
	=
	A \sin \left( \kappa r \right)
	+
	\underbrace{\cancel{B \cos \left( \kappa r \right)}}_{\equiv 0 \text{ when } r \to \infty}
\end{equation}
\begin{equation} \label{eq:2.23}
	u_{\text{II.}}
	=
	C e^{-kr}
	+
	\underbrace{\cancel{D e^{kr}}}_{\equiv 0 \text{ when } r = 0}
\end{equation}
Also at the $r=b$ point both solutions, also their derivates should be equal:

\begin{equation} \label{eq:2.24}
	A \sin \left( \kappa b \right)
	=
	c * e^{-kb}
\end{equation}
\begin{equation} \label{eq:2.25}
	A \kappa \cos \left( \kappa b \right)
	=
	- k * c * e^{-kb}
\end{equation}
Dividing Eq. \eqref{eq:2.25} by Eq. \eqref{eq:2.24} we get

\begin{equation} \label{eq:2.26}
	\kappa \cot \left( \kappa b \right)
	=
	-k
\end{equation}
If $\left| E \right| << V_{0}$, then $k \approx 0$ in Eq. \eqref{eq:2.26} since

\begin{equation*}
	E
	=
	\frac{k^{2} \hbar^{2}}{2 \mu}
\end{equation*}
This makes

\begin{equation} \label{eq:2.27}
	\kappa \cot \left( \kappa b \right)
	=
	0
	\quad \longrightarrow \quad
	\kappa b
	=
	\frac{\pi}{2} \left( 2n + 1 \right)
\end{equation}
where $n \in \mathbb{N}_{0}$. Using the value of $\kappa$ from Eq. \eqref{eq:2.18}, we can write the following equation:

\begin{equation} \label{eq:2.28}
	\kappa^{2} b^{2}
	=
	\frac{2 \mu}{\hbar^{2}} \left( V_{0} + E \right) * b^{2}
	=
	\frac{\pi^{2}}{4} \left( 2n + 1 \right)^{2}
\end{equation}
Where $E \approx 0$. Thus the value of $V_{0}^{\left( n \right)}$ could be approximated. The maximal distance of the strong interaction between nucleons is $r_{max} \approx 1.4\ \text{fm}$, so we'll use this value for the width $b$ of the potential well. Using this the approximation becomes

\begin{equation} \label{eq:2.29}
	V_{0}^{\left( n \right)}
	=
	\frac{\hbar^{2}}{2 \mu} \frac{\pi^{2}}{4 b^{2}} \left( 2n + 1 \right)^{2}
	=
	V_{0}^{\left( 0 \right)} \left( 2n + 1 \right)^{2}
\end{equation}
Where

\begin{equation} \label{eq:2.30}
	V_{0}^{\left( 0 \right)}
	\approx
	50 MeV
\end{equation}
Which means in the ground state, there is an $\approx 50\ \text{MeV}$ deep potential well in the deuteron, with only one, $E \approx 0$ bound state. For $n=1$ the potential well is $9$ times deeper, and there are two bound states present. For the $l=1$ case the ground state potential is $4$ times deeper.

\subsection{Wave function of two-nucleon system}
The wave function of the system containing two interacting nucleons could be decomposed into a $\varphi_{L}$ \textbf{spatial}, a $\chi_{S}$ \textbf{spin} and a $\xi_{T}$ \textbf{isospin} part:

\begin{equation} \label{eq:2.31}
	\Psi \left( \boldsymbol{1}; \boldsymbol{2} \right)
	=
	\varphi_{L} \left( \boldsymbol{1}; \boldsymbol{2} \right)
	*
	\chi_{S} \left( \boldsymbol{1}; \boldsymbol{2} \right)
	*
	\xi_{T} \left( \boldsymbol{1}; \boldsymbol{2} \right)
\end{equation}
Where the arguments $\boldsymbol{1}$ and $\boldsymbol{2}$ denotes the state of the first and second nucleon respectively:

\begin{equation} \label{eq:2.32}
	\Psi \left( \boldsymbol{1}; \boldsymbol{2} \right)
	=
	\Psi \left( \vec{r}_{1}, \vec{s}_{1}, \vec{\tau}_{1}; \vec{r}_{2}, \vec{s}_{2}, \vec{\tau}_{2} \right)
\end{equation}
\begin{equation} \label{eq:2.33}
	\varphi_{L} \left( \boldsymbol{1}; \boldsymbol{2} \right)
	=
	\varphi_{L} \left( \vec{r}_{1}; \vec{r}_{2} \right)
	\equiv
	\varphi_{L} \left( \vec{r}_{1} - \vec{r}_{2} \right)
\end{equation}
\begin{equation} \label{eq:2.34}
	\chi_{S} \left( \boldsymbol{1}; \boldsymbol{2} \right)
	=
	\chi_{S} \left( \vec{s}_{1}; \vec{s}_{2} \right)
\end{equation}
\begin{equation} \label{eq:2.35}
	\xi_{T} \left( \boldsymbol{1}; \boldsymbol{2} \right)
	=
	\xi_{T} \left( \vec{\tau}_{1}; \vec{\tau}_{2} \right)
\end{equation}
Speaking about a fermionic system of nucleons, for the inversion of the two particles the wave function changes its sign:

\begin{equation} \label{eq:2.36}
	\Psi \left( \boldsymbol{1}; \boldsymbol{2} \right)
	=
	- \Psi \left( \boldsymbol{2}; \boldsymbol{1} \right)
\end{equation}
This also means, that the product of the parity of the spatial, spin and isospin parts should be $-1$:

\begin{equation} \label{eq:2.37}
	\Pi_{\varphi} * \Pi_{\chi} * \Pi_{\xi}
	=
	-1
\end{equation}
We need to further discuss the spin and isospin parts to have a better understanding, what they mean and how are they constructed. We define the two-nucleon spin and isospin components as the following:

\begin{equation} \label{eq:2.38}
	\chi_{S,\nu} \left( \boldsymbol{1}; \boldsymbol{2} \right)
	=
	\sum_{\nu_{1}, \nu_{2}} \Braket{s_{1}, \nu_{1}, s_{2}, \nu_{2}|S, \nu}
	\chi_{s_{1}, \nu_{1}} \left( \boldsymbol{1} \right) \chi_{s_{2}, \nu_{2}} \left( \boldsymbol{2} \right)
\end{equation}
\begin{equation} \label{eq:2.39}
	\xi_{T,\tau} \left( \boldsymbol{1}; \boldsymbol{2} \right)
	=
	\sum_{t_{1}, t_{2}} \Braket{t_{1}, \tau_{1}, t_{2}, \tau_{2}|T, \tau}
	\xi_{t_{1}, \tau_{1}} \left( \boldsymbol{1} \right) \xi_{t_{2}, \tau_{2}} \left( \boldsymbol{2} \right)
\end{equation}
Here $\vec{S} = \sum_{i} \vec{s}_{i}$ is the \textbf{total spin vector}, where $\vec{s}_{i}$ is the spin vector of the individual particles. The term $S$ in the equations above is the norm of the total spin vector $|| \vec{S} || = S$, called \textbf{total spin}. The index $\nu$ (sometimes written as $m_{s}$, meaning magnetic spin quantum number) denotes its corresponding \textbf{secondary spin quantum numbers} ranging from $-S$ to $+S$ in steps of one. This spans $2S + 1$ different values of $\nu$. Similarly, the $s_{i}$ and $\nu_{i}$ terms correspond to the spin and secondary spin quantum number of the two nucleons. 

\subsection{Possible two-nucleon states}
We can write Eq. \eqref{eq:2.38} and Eq. \eqref{eq:2.39} with another common notation for eg. the spin component as

\begin{equation} \label{eq:2.40}
	\Ket{S, \nu}
	=
	\sum_{\nu_{1}+\nu_{2} = \nu}
	C_{\nu_{1}, \nu_{2}, \nu}^{s_{1}, s_{2}, S} \Ket{s_{1}, \nu_{1}} \Ket{s_{2}, \nu_{2}}
\end{equation}
Similar equation could be formalised for the isospin component too. In Eq. \eqref{eq:2.38} and Eq. \eqref{eq:2.39} the $\Braket{\dots | \dots}$ is equivalent to the $C$ coefficient in Eq. \eqref{eq:2.40}, and which are called the Clebsch-Gordon coefficients. \par
A nucleon is a particle with a spin of magnitude $\frac{1}{2} \hbar$, or using the $\hbar = 1$ metric its $\frac{1}{2}$. Thus means in a system of two nucleons the spins are equally

\begin{equation*}
	s_{1} = s_{2} = \frac{1}{2}
\end{equation*}
Furthermore, because the total spin of a system of two nucleons should be

\begin{equation} \label{eq:2.41}
	|| \vec{s}_{1} - \vec{s}_{2} || \leq || \vec{S} || \leq || \vec{s}_{1} + \vec{s}_{2} ||
\end{equation}
the value of $S$ could be either $0$ or $1$, which generates the singlet and triplet state respectively. In the singlet state there are only $1$ possible value of $\nu$, while in the triplet state there are $3$ different ones. Using the discussed relation between $s_{i}$ and $\nu_{i}$ values, we can declare the possible values for $\Ket{S, \nu}$ by substituting is the $s_{i}$, $\nu_{i}$ values in Eq. \eqref{eq:2.10}:

\begin{equation} \label{eq:2.42}
	\Ket{s_{1}, \nu_{1}} \Ket{s_{2}, \nu_{2}}
	=
	\Ket{\frac{1}{2}, +\frac{1}{2}} \Ket{\frac{1}{2}, +\frac{1}{2}}
	\to
	\left( \uparrow \uparrow \right)
\end{equation}
\begin{equation} \label{eq:2.43}
	\Ket{s_{1}, \nu_{1}} \Ket{s_{2}, \nu_{2}}
	=
	\Ket{\frac{1}{2}, -\frac{1}{2}} \Ket{\frac{1}{2}, +\frac{1}{2}}
	\to
	\left( \downarrow \uparrow \right)
\end{equation}
\begin{equation} \label{eq:2.44}
	\Ket{s_{1}, \nu_{1}} \Ket{s_{2}, \nu_{2}}
	=
	\Ket{\frac{1}{2}, +\frac{1}{2}} \Ket{\frac{1}{2}, -\frac{1}{2}}
	\to
	\left( \uparrow \downarrow \right)
\end{equation}
\begin{equation} \label{eq:2.45}
	\Ket{s_{1}, \nu_{1}} \Ket{s_{2}, \nu_{2}}
	=
	\Ket{\frac{1}{2}, -\frac{1}{2}} \Ket{\frac{1}{2}, -\frac{1}{2}}
	\to
	\left( \downarrow \downarrow \right)
\end{equation}
Since $S \in \left\{ 0, 1 \right\}$ and $\nu = \nu_{1} + \nu_{2}$, while $\nu = -S, -S+1, \dots, S-1, S$, the singlet and triplet states could be defined:

\begin{equation} \label{eq:2.46}
	S = 0
	\quad \longrightarrow \quad
	\Ket{S, \nu} = \Ket{0,0} = \frac{1}{\sqrt{2}} \left( \uparrow \downarrow - \downarrow \uparrow \right)
	\quad \quad
	\text{(Singlet)}
\end{equation}
\begin{equation} \label{eq:2.47}
	S = 1
	\quad \longrightarrow \quad
	\begin{cases}	
		\Ket{S, \nu} = \Ket{1,1} = \uparrow \uparrow \\
		\Ket{S, \nu} = \Ket{1,0} = \frac{1}{\sqrt{2}} \left( \uparrow \downarrow + \downarrow \uparrow \right) \\
		\Ket{S, \nu} = \Ket{1,-1} = \downarrow \downarrow
	\end{cases}	
	\quad \quad
	\text{(Triplet)}
\end{equation}
Inverting the two particles, in the case of the triplet state the wave function remains the same, but in the case of the singlet state it changes its sign. It could be easily seen by changing $\uparrow$ and $\downarrow$ in Eq. \eqref{eq:2.46} and Eq. \eqref{eq:2.47}:

\begin{equation} \label{eq:2.48}
	\Pi_{\chi} \left( S=0 \right) = -1
	\quad ; \quad
	\Pi_{\chi} \left( S=1 \right) = +1
\end{equation}